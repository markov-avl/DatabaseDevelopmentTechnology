\section{Тестирование}\label{sec:testing}


\renewcommand{\img}[3]{
    \begin{figure}[H]
        \center{\includegraphics[scale=#2]{tests/#1}}
        \caption{#3}
        \label{fig:#1}
    \end{figure}
}
\newcommand{\test}[3]{
    #3 (рис. \ref{fig:#2}):
    \img{#2}{0.5}{#1}
}


\test{Тестирование пустого списка тарифов}{test-1}{Если в базе данных не будет ни одной записи о тарифе, то все списки с тарифами будут пустыми}

\test{Пустой пароль и логин пользователя}{test-2}{Если при создании пользователя не ввести в логин и пароль ничего, то такого пользователя создать такого нельзя будет}

\test{Более 32-х символов для логина и пароля пользователя}{test-3}{Если при создании пользователя ввести более 32 символов в логине и пароле, то создать такого пользователя нельзя будет (разрешаются любые символы кроме пробелов -- они совершенно не могут вводиться)}

Если в поиске пользователя вводить символы, не входящие в логин пользователя, то такой пользователь уходит из списка (рис. \ref{fig:test-4-1}-\ref{fig:test-4-2}).
\img{test-4-1}{0.5}{Список всех пользователей}
\img{test-4-2}{0.5}{Список всех пользователей, в логин которого входит <<ana>>}

Если в поиске пользователя выбирать конкретную должность, то пользователи другой должности не будут попадать в список пользователей (рис. \ref{fig:test-5-1}-\ref{fig:test-5-2}).
\img{test-5-1}{0.5}{Список всех пользователей}
\img{test-5-2}{0.5}{Список всех пользователей-менеджеров}
\img{test-5-3}{0.5}{Список всех пользователей, должность которых <<продавец-консультант>> и в логин которых входит <<ana>>}

\test{Пустые поля для создания тарифа}{test-6}{Если при создании тарифа не ввести ничего, то такой тариф создать нельзя будет}

Поле <<Название тарифа>> позволяет вводить туда любые строки до 64 символов, которые не оканчиваются на пробел (рис. \ref{fig:test-7-1}-\ref{fig:test-7-3}).
\img{test-7-1}{0.5}{Валидное название тарифа}
\img{test-7-2}{0.5}{В названии тарифа более 64 символов}
\img{test-7-3}{0.5}{В названии тарифа в конце пробельный символ}

Поле <<Абонентская плата>> позволяет вводить туда целые числа от 150 до 5000 (рис. \ref{fig:test-8-1}-\ref{fig:test-8-4}).
\img{test-8-1}{0.5}{Левое крайнее значение валидной абонентской платы}
\img{test-8-2}{0.5}{Правое крайнее значение валидной абонентской платы}
\img{test-8-3}{0.5}{Левое крайнее значение невалидной абонентской платы}
\img{test-8-4}{0.5}{Правое крайнее значение невалидной абонентской платы}

Поле <<Интернет трафик>> позволяет вводить туда нецелые числа от 0 до 50, а также -1 (рис. \ref{fig:test-9-1}-\ref{fig:test-9-5}).
\img{test-9-1}{0.5}{Ввод безлимита для интернет трафика}
\img{test-9-2}{0.5}{Левое крайнее значение валидного интернет трафика}
\img{test-9-3}{0.5}{Правое крайнее значение валидного интернет трафика}
\img{test-9-4}{0.5}{Левое крайнее значение невалидного интернет трафика}
\img{test-9-5}{0.5}{Правое крайнее значение невалидного интернет трафика}

Поле <<Минуты>> позволяет вводить туда целые числа от 0 до 2000, а также -1 (рис. \ref{fig:test-10-1}-\ref{fig:test-10-5}).
\img{test-10-1}{0.5}{Ввод безлимита для количества минут}
\img{test-10-2}{0.5}{Левое крайнее значение валидного количества минут}
\img{test-10-3}{0.5}{Правое крайнее значение валидного количества минут}
\img{test-10-4}{0.5}{Левое крайнее значение невалидного количества минут}
\img{test-10-5}{0.5}{Правое крайнее значение невалидного количества минут}

Поле <<SMS>> позволяет вводить туда целые числа от 0 до 2000, а также -1 (рис. \ref{fig:test-11-1}-\ref{fig:test-11-5}).
\img{test-11-1}{0.5}{Ввод безлимита для количества SMS}
\img{test-11-2}{0.5}{Левое крайнее значение валидного количества SMS}
\img{test-11-3}{0.5}{Правое крайнее значение валидного количества SMS}
\img{test-11-4}{0.5}{Левое крайнее значение невалидного количества SMS}
\img{test-11-5}{0.5}{Правое крайнее значение невалидного количества SMS}