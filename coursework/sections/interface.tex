\section{Интерфейс между пользователем и БД}\label{sec:interface}


\renewcommand{\img}[3]{
    \begin{figure}[H]
        \center{\includegraphics[scale=#2]{screenshots/#1}}
        \caption{#3}
        \label{fig:#1}
    \end{figure}
}


\subsection{Описание форм}

В данной главе описываются все экранные формы.

На главной странице (рис. \ref{fig:main-form}) представлена экранная форма, на которой можно просмотреть все актуальные тарифы сотового оператора. Слева в верхнем углу присутствует ссылка <<Тарифы>> на главную форму, которая всегда будет доступна во всех остальных формах. Также на главной странице в правом верхнем углу присутствует ссылка на форму авторизации в системе.
\img{main-form}{0.49}{Главная страница}

Если перейти к авторизации в системе, то появится следующая форма с полями для ввода логина и пароля пользователя (рис. \ref{fig:auth-form}).
\img{auth-form}{0.49}{Форма авторизации}

После успешной авторизации в системе, на главной странице вместо ссылки на форму <<Авторизация>> появляется ссылка на форму <<Личный кабинет>> (рис. \ref{fig:account-ref}).
\img{account-ref}{0.49}{Ссылка на личный кабинет}

В личном кабинете (или <<панели>>) каждого пользователя в верхнем правом углу присутствуют кнопка <<Выйти>> (рис. \ref{fig:quit-button}) для выхода из аккаунта данного пользователя. После нажатия данной кнопки автоматически происходит переход на главную страницу.
\img{quit-button}{0.49}{Ссылка на личный кабинет}

% Администратор
Если авторизоваться в системе как администратор, то произойдёт переход к следующей форме -- панель взаимодействия с системой для администратора (рис. \ref{fig:admin-menu-form}). На данной странице администратор может создавать пользователей, а также редактировать и удалять уже существующих.
\img{admin-menu-form}{0.49}{Панель администратора}

Секция <<Создать пользователя>> (рис. \ref{fig:admin-add-users}) позволяет администратору создать пользователя с логином и паролем, а также назначить новому пользователю должность (рис. \ref{fig:admin-add-users-types}).
\img{admin-add-users}{0.49}{Секция <<Создать пользователя>>}
\img{admin-add-users-types}{0.49}{Выпадающий список с должностями пользователей}

Секции <<Поиск>> и <<Пользователи>> (рис. \ref{fig:admin-edit-users}) позволяют администратору редактировать и удалять пользователей. Администратору представлены все пользователи системы, список которых можно сужать с помощью поиска.
\img{admin-edit-users}{0.49}{Секции <<Поиск>> и <<Пользователи>>}
\img{admin-edit-users-types}{0.49}{Выпадающий список с должностями пользователей}

Если нажать на кнопку <<Изменить>> напротив пользователя, появляется следующее модальное окно (рис. \ref{fig:admin-edit-users-editing}), в котором можно изменить логин, пароль и должность данного пользователя.
\img{admin-edit-users-editing}{0.49}{Модальное окно <<Изменение данных пользователя>>}

% Менеджер
Если авторизоваться в системе как менеджер, то произойдет переход к следующей форме -- панель взаимодействия с системой для менеджера (рис. \ref{fig:manager-menu-form}). На данной странице менеджер может создавать новые тарифы, а также редактировать и удалять уже существующие.
\img{manager-menu-form}{0.49}{Панель менеджера}

Секция <<Создать тариф>> (рис. \ref{fig:manager-add-tariff}) позволяет менеджеру создать тариф, указав его название, абонентскую плату, интернет трафик, количество минут и количество SMS.
\img{manager-add-tariff}{0.49}{Секция <<Создать тариф>>}

Секции <<Поиск>> и <<Тарифы>> (рис. \ref{fig:manager-edit-tariffs}) позволяют менеджеру редактировать и удалять тарифы. Менеджеру представлены все тарифы системы, список которых можно сужать с помощью поиска.
\img{manager-edit-tariffs}{0.49}{Секции <<Поиск>> и <<Тарифы>>}

% Продавец-консультант
Если авторизоваться в системе как продавец-консультант, то произойдет переход к следующей форме -- панель взаимодействия с системой для продавца-консультанта (рис. \ref{fig:shop-assistant-menu-form}). На данной странице продавец-консультант может проверять актуальность данных клиента (редактировать данные клиента, если данные устарели), добавлять новых абонентов, а также удалять существующих абонентов.
\img{shop-assistant-menu-form}{0.49}{Панель продавца-консультанта}

Секция <<Проверка актуальности данных клиента>> (рис. \ref{fig:shop-assistant-client-check}) позволяет продавцу-консультанту проверять на актуальность паспортные данные клиента (если такой существует).
\img{shop-assistant-client-check}{0.49}{Секция <<Проверка актуальности данных клиента>>}

Если серия и номер паспорта клиента будут найдены в системе, то напротив клиента появится кнопка для изменения паспортных данных клиента (рис. \ref{fig:shop-assistant-client-found}).
\img{shop-assistant-client-found}{0.49}{Секция <<Проверка актуальности данных клиента>> с найденным клиентом}

Если нажать на кнопку <<Изменить>> напротив клиента, появляется следующее модальное окно (рис. \ref{fig:shop-assistant-client-editing}), в котором можно изменить его серию и номер паспорта, ФИО и место прописки.
\img{shop-assistant-client-editing}{0.49}{Модальное окно <<Изменение клиента>>}

Секция <<Добавление абонента>> (рис. \ref{fig:shop-assistant-add-subscriber}) позволяет продавцу-консультанту добавлять нового абонента.
\img{shop-assistant-add-subscriber}{0.49}{Секция <<Добавление абонента>>}

Для набора даты рождения абонента (клиента) можно открыть специальное окно для выбора даты (рис. \ref{fig:shop-assistant-date}).
\img{shop-assistant-date}{0.49}{Cпециальное окно для выбора даты}

Если нажать на кнопку <<Добавить номер>>, появляется следующее модальное окно (рис. \ref{fig:shop-assistant-phones}), в котором можно выбрать свободный телефонный номер для нового абонента.
\img{shop-assistant-phones}{0.49}{Модальное окно <<Список доступных номеров>>}

Секция <<Абоненты клиента>> (рис. \ref{fig:shop-assistant-remove-subscriber}) позволяет продавцу-консультанту после набора серии и номера паспорта клиента удалить абонента, зарегистрированного на данного клиента (если такие имеются).
\img{shop-assistant-remove-subscriber}{0.49}{Секция <<Абоненты клиента>>}

Если найдутся абоненты, зарегистрированные на данного клиента, то появляется кнопка <<Удалить>> (рис. \ref{fig:shop-assistant-subscriber-found}) напротив абонентского счёта).
\img{shop-assistant-subscriber-found}{0.49}{Секция <<Абоненты клиента>> с найденным абонентом}

% Абонент
Если авторизоваться в системе как абонент, то произойдет переход к следующей форме -- панель взаимодействия с системой для абонента (рис. \ref{fig:subscriber-menu-form}). На данной странице абонент может просмотреть краткую информацию о себе, изменить свой пароль, пополнить баланс и изменить тариф.
\img{subscriber-menu-form}{0.49}{Панель абонента}

Секция <<Изменить пароль>> (рис. \ref{fig:subscriber-change-password}) позволяет абоненту изменить его пароль для входа в систему.
\img{subscriber-change-password}{0.49}{Секция <<Изменить пароль>>}

Секция <<Пополнить баланс>> (рис. \ref{fig:subscriber-add-balance}) позволяет абоненту пополнить его баланс на определённую сумму.
\img{subscriber-add-balance}{0.49}{Секция <<Пополнить баланс>>}

Секция <<Изменить тариф>> (рис. \ref{fig:subscriber-change-tariff}) позволяет абоненту изменить его текущий тариф.
\img{subscriber-change-tariff}{0.49}{Секция <<Изменить тариф>>}

Если нажать на кнопку <<Выбрать>> напротив названия тарифа, появляется следующее модальное окно (рис. \ref{fig:subscriber-tariffs}), в котором можно выбрать новый тариф из всего списка тарифов.
\img{subscriber-tariffs}{0.49}{Модальное окно <<Тарифы>>}


\subsection{Описание запросов}


\renewcommand{\arraystretch}{1.5}
\begin{xltabular}[h]{\textwidth}{|p{0.7 \textwidth}|X|}
    \caption{Описание запросов} \\
    \hline
    \textbf{Запрос}                                                                                                                                                                                                       & \textbf{Для чего используется}                                                  \\
    \hline \endhead
    \texttt{SELECT * FROM tariffs}                                                                                                                                                                                        & Получение всей информации о каждом тарифе в системе                             \\ \hline
    \texttt{SELECT login, password, user\_type FROM users INNER JOIN user\_types ON id=user\_type\_id WHERE login='\{логин\}' AND password='\{пароль\}' LIMIT 1}                                                          & Получение данных авторизации пользователя                                       \\ \hline
    \texttt{INSERT INTO tariffs (name, subscription\_fee, internet\_traffic, minutes, sms) VALUES ('\{название\}', \{абонентская плата\}, \{интернет трафик\}, \{количество минут\}, \{количество SMS\})}                 & Добавление нового тарифа в систему                                              \\ \hline
    \texttt{UPDATE tariffs SET \{атрибут 1\}=\{новое значение атрибута 1\}, ..., \{атрибут n\}=\{новое значение атрибута n\} WHERE name='\{название\}'}                                                                   & Обновление данных о тарифе                                                      \\ \hline
    \texttt{SELECT COUNT(*) as count FROM clients WHERE passport=\{серия и номер паспорта\} LIMIT 1}                                                                                                                      & Проверка, существует ли клиент с данной серией и номером паспорта               \\ \hline
    \texttt{UPDATE clients SET \{атрибут 1\}=\{новое значение атрибута 1\}, ..., \{атрибут n\}=\{новое значение атрибута n\} WHERE passport='\{серия и номер паспорта\}'}                                                 & Обновление данных о клиенте                                                     \\ \hline
    \texttt{SELECT id FROM places\_of\_registration WHERE place\_of\_registration=\{место прописки\} LIMIT 1}                                                                                                             & Получение кода данного места прописки, если оно существует                      \\ \hline
    \texttt{INSERT INTO places\_of\_registration (place\_of\_registration) values ('\{место прописки\}')}                                                                                                                 & Добавление места прописки в систему                                             \\ \hline
    \texttt{INSERT INTO clients (passport, full\_name, place\_of\_registration\_id, date\_of\_birth) VALUES (\{серия и номер паспорта\}, '\{ФИО\}', \{код места прописки\}, TO\_DATE('\{дата рождения\}', 'dd.mm.yyyy'))} & Добавление клиента в систему                                                    \\ \hline
    \texttt{SELECT COUNT(*) as count FROM subscribers WHERE passport=\{серия и номер паспорта\} AND balance<0}                                                                                                            & Проверка, нет ли задолженности у клиента ни по одному из его абонентских счетов \\ \hline
    \texttt{SELECT COUNT(*) as count FROM subscribers WHERE passport=\{серия и номер паспорта\}}                                                                                                                          & Получение количества зарегистрированных на клиента абонентских счетов           \\ \hline
    \texttt{INSERT INTO subscribers (account, passport) VALUES ('\{абонентский счёт\}', \{серия и номер паспорта\})}                                                                                                      & Добавление абонента в систему                                                   \\ \hline
    \texttt{UPDATE phone\_numbers SET subscriber\_account='\{абонентский счёт\}' WHERE phone\_number=\{телефонный номер\}}                                                                                                & Регистрация абонентского счёта на данный телефонный номер                       \\ \hline
    \texttt{INSERT INTO users (login, password, user\_type\_id) VALUES ('\{телефонный номер\}', '\{абонентский счёт\}', (SELECT id FROM user\_types WHERE user\_type='Абонент'))}                                         & Добавление данных авторизации абонента по умолчанию                             \\ \hline
    \texttt{SELECT COUNT(*) as count FROM subscribers WHERE account='\{абонентский счёт\}' AND balance<0}                                                                                                                 & Проверка, нет ли задолженности по данному абонентскому счёту                    \\ \hline
    \texttt{DELETE FROM subscribers WHERE account='\{абонентский счёт\}'}                                                                                                                                                 & Удаление абонента из системы                                                    \\ \hline
    \texttt{DELETE FROM users WHERE login='\{телефонный номер\}'}                                                                                                                                                         & Удаление данных авторизации абонента                                            \\ \hline
    \texttt{SELECT phone\_number FROM phone\_numbers WHERE subscriber\_account IS NULL}                                                                                                                                   & Получение всех свободных телефонных номеров сотового оператора                  \\ \hline
    \texttt{UPDATE subscribers SET connected\_tariff='\{название тарифа\}'  WHERE account='\{абонентский счёт\}'}                                                                                                         & Подключение данного тарифа на определённый абонентский счёт                     \\ \hline
    \texttt{SELECT COUNT(*) as count FROM subscribers INNER JOIN tariffs ON name='\{название тарифа\}' AND balance>=subscription\_fee WHERE account='\{абонентский счёт\}'}                                               & Проверка, хватает ли денег на балансе для подключения данного тарифа            \\ \hline
    \texttt{SELECT account, balance, phone\_number, connected\_tariff FROM subscribers INNER JOIN phone\_numbers ON subscriber\_account=account WHERE account='\{абонентский счёт\}'}                                     & Получение данных об абоненте с данным абонентским счётом                        \\ \hline
\end{xltabular}


\subsection{Описание диалогов}


При попытке неверно ввести логин и пароль появляется следующее диалоговое окно (рис. \ref{fig:failed-auth}):
\img{failed-auth}{0.49}{Диалог при неверном вводе логина или пароля в систему}

При попытке активации действия, для которого нужны корректно введённые данные в каждом поле некоторой секции, появляется следующее диалоговое окно (рис. \ref{fig:failed-action}):
\img{failed-action}{0.49}{Диалог при неудачной попытки активации действия}

Будучи авторизованным как администратор и при попытке создать пользователя с уже существующим логином появляется следующее диалоговое окно (рис. \ref{fig:failed-add-user}):
\img{failed-add-user}{0.49}{Диалог при неудачной попытки создания пользователя}

Будучи авторизованным как администратор и при успешном изменении пользователя появляется следующее диалоговое окно (рис. \ref{fig:edit-user-success}):
\img{edit-user-success}{0.49}{Диалог при успешном изменении пользователя}

Будучи авторизованным как администратор и при изменении данных пользователя на те же появляется следующее диалоговое окно (рис. \ref{fig:user-no-changes}):
\img{user-no-changes}{0.49}{Диалог при изменении данных пользователя на те же}

Будучи авторизованным как администратор и при изменении логина пользователя на логин другого пользователя появляется следующее диалоговое окно (рис. \ref{fig:user-login-found}):
\img{user-login-found}{0.49}{Диалог при изменении логина пользователя на логин другого пользователя}

Будучи авторизованным как администратор и при удалении пользователя (в данном случае пользователя с логином <<manager1>>) появляется следующее диалоговое окно (рис. \ref{fig:remove-user}):
\img{remove-user}{0.49}{Диалог при удалении пользователя}

Будучи авторизованным как менеджер и при попытке добавить тариф с уже существующим названием (в данном случае <<Всегда в сети>>) появится следующее диалоговое окно (рис. \ref{fig:invalid-tariff-name-key}):
\img{invalid-tariff-name-key}{0.49}{Диалог при добавлении существующего тарифа в систему}

Будучи авторизованным как менеджер и при успешном изменении данных тарифа появляется следующее диалоговое окно (рис. \ref{fig:tariff-changed}):
\img{tariff-changed}{0.49}{Диалог при успешном изменении данных тарифа}

Будучи авторизованным как менеджер и при изменении данных тарифа на те же появляется следующее диалоговое окно (рис. \ref{fig:tariff-no-changes}):
\img{tariff-no-changes}{0.49}{Диалог при изменении данных тарифа на те же}

Будучи авторизованным как менеджер и при удалении тарифа (в данном случае <<Новый тариф>>) появится следующее диалоговое окно (рис. \ref{fig:tariff-removed}):
\img{tariff-removed}{0.49}{Диалог при удалении тарифа}

Будучи авторизованным как продавец-консультант и при успешном изменении данных о клиенте появляется следующее диалоговое окно (рис. \ref{fig:client-changed}):
\img{client-changed}{0.49}{Диалог при успешном изменении данных о клиенте в системе}

Будучи авторизованным как продавец-консультант и при изменении данных клиента на те же появляется следующее диалоговое окно (рис. \ref{fig:client-no-changes}):
\img{client-no-changes}{0.49}{Диалог при изменении данных клиента на те же}

Будучи авторизованным как продавец-консультант и при добавлении абонента, если не выбрать для него телефонный номер, появляется следующее диалоговое окно (рис. \ref{fig:phone-not-selected}):
\img{phone-not-selected}{0.49}{Диалог при добавлении абонента, если не выбрать для него телефонный номер}

Будучи авторизованным как продавец-консультант и при попытке добавления абонента, данные которого уже зарегистрированы на 5 абонентских счетов, появляется следующее диалоговое окно (рис. \ref{fig:too-many-accounts}):
\img{too-many-accounts}{0.49}{Диалог при добавлении абонента, данные которого уже зарегистрированы на 5 абонентских счетов}

Будучи авторизованным как продавец-консультант и при попытке добавления абонента, по абонентским счетам которого имеется хотя бы одна задолженность, появляется следующее диалоговое окно (рис. \ref{fig:subscriber-debts}):
\img{subscriber-debts}{0.49}{Диалог при добавлении абонента, по абонентским счетам которого имеется хотя бы одна задолженность}

Будучи авторизованным как продавец-консультант и при попытке добавления абонента с уже существующим абонентским счетом появляется следующее диалоговое окно (рис. \ref{fig:invalid-subscriber-account}):
\img{invalid-subscriber-account}{0.49}{Диалог при абонента с уже существующим абонентским счетом}

Будучи авторизованным как продавец-консультант и при успешном добавлении абонента появляется следующее диалоговое окно (рис. \ref{fig:success-subscriber-adding}):
\img{success-subscriber-adding}{0.49}{Диалог при успешном добавлении абонента в систему}

Будучи авторизованным как продавец-консультант и при успешной попытке удаления абонента появляется следующее диалоговое окно (рис. \ref{fig:subscriber-removed}):
\img{subscriber-removed}{0.49}{Диалог при успешном удалении абонента}

Будучи авторизованным как продавец-консультант и при попытке удаления абонента, по абонентскому счёту которого имеется задолженность, появляется следующее диалоговое окно (рис. \ref{fig:subscriber-account-debt}):
\img{subscriber-account-debt}{0.49}{Диалог при удалении абонента, по абонентскому счёту которого имеется задолженность}

Будучи авторизованным как абонент и при успешном изменении пароля появляется следующее диалоговое окно (рис. \ref{fig:password-changed}):
\img{password-changed}{0.49}{Диалог при успешном изменении пароля}

Будучи авторизованным как абонент и при изменении пароля на тот же появляется следующее диалоговое окно (рис. \ref{fig:password-no-changes}):
\img{password-no-changes}{0.49}{Диалог при изменении пароля на тот же}

Будучи авторизованным как абонент и при успешном пополнении баланса появляется следующее диалоговое окно (рис. \ref{fig:balance-changed}):
\img{balance-changed}{0.49}{Диалог при успешном пополнении баланса}

Будучи авторизованным как абонент и при неудачной попытки пополнения баланса (если счёт на балансе составляет -1 руб.) появляется следующее диалоговое окно (рис. \ref{fig:balance-adding-failed}):
\img{balance-adding-failed}{0.49}{Диалог при неудачной попытки пополнения баланса}

Будучи авторизованным как абонент и при успешном подключении тарифа появляется следующее диалоговое окно (рис. \ref{fig:success-tariff-connecting}):
\img{success-tariff-connecting}{0.49}{Диалог при успешном подключении тарифа}