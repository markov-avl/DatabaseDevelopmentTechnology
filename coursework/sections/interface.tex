\section{Интерфейс между пользователем и БД}\label{sec:interface}


\renewcommand{\img}[3]{
    \begin{figure}[H]
        \center{\includegraphics[scale=#2]{screenshots/#1}}
        \caption{#3}
        \label{fig:#1}
    \end{figure}
}


\subsection{Описание форм}

В данной главе описываются все экранные формы. Все формы имеют кнопки для возвращения на предыдущую форму (кроме главной страницы), поэтому в дальнейшем на этом не будет заостряться внимание.

На главной странице (рис. \ref{fig:main-form}) представлена экранная форма, позволяющая перейти к просмотру всех актуальных тарифов сотового оператора или же к авторизации в системе.
\img{main-form}{0.5}{Главная страница}

Если перейти к авторизации в системе, то появится следующая форма с полями для ввода логина и пароля пользователя (рис. \ref{fig:auth-form}).
\img{auth-form}{0.5}{Форма авторизации}

Если авторизоваться в системе как менеджер, то произойдет переход к следующей форме -- меню взаимодействия с системой для менеджера (рис. \ref{fig:manager-menu-form}). На данной странице менеджер может выбрать 2 возможных типа взаимодействия: добавить новый тариф или изменить уже существующий.
\img{manager-menu-form}{0.5}{Меню взаимодействия с системой для менеджера}

На странице добавления тарифа (рис. \ref{fig:tariff-adding-form}) находятся следующие формы для заполнения: название, абонентская плата, интернет трафик, количество минут и количество SMS, а также кнопка для добавления будущего тарифа в систему.
\img{tariff-adding-form}{0.5}{Форма добавления тарифа}

На странице обновления данных о тарифе (рис. \ref{fig:tariff-editing-form}) находится список всех существующих тарифов, атрибуты каждого из которых можно изменять и сохранять с помощью рядом расположенной кнопки. Также на странице расположен поиск по названию тарифа.
\img{tariff-editing-form}{0.5}{Форма обновления данных о тарифе}

Если авторизоваться в системе как продавец-консультант, то произойдет переход к следующей форме -- меню взаимодействия с системой для продавца-консультанта (рис. \ref{fig:shop-assistant-menu-form}). На данной странице продавец-консультант может выбрать 4 возможных типа взаимодействия: обновить данные о клиенте, добавить абонента, удалить абонента и просмотреть свободные телефонные номера сотового оператора.
\img{shop-assistant-menu-form}{0.5}{Меню взаимодействия с системой для продавца-консультанта}

На странице обновления данных о клиенте (рис. \ref{fig:client-editing-form-1}) можно ввести несколько серий и номеров паспорта клиента для проверки, существует ли хотя бы один из них в системе. Если такой находится, то появляется возможность изменения других данных о клиенте (рис. \ref{fig:client-editing-form-2}): серия и номер паспорта, ФИО, дата рождения и место прописки.
\img{client-editing-form-1}{0.5}{Форма проверки существования клиента в системе}
\img{client-editing-form-2}{0.5}{Форма обновления данных о клиенте}

На странице добавления абонента (рис. \ref{fig:subscriber-adding-form}) находятся следующие формы для заполнения: абонентский счёт, телефонный номер, серия и номер паспорта и остальные данные о клиенте (если клиент не найден в системе), а также кнопка для добавления будущего абонента в систему.
\img{subscriber-adding-form}{0.5}{Форма добавления абонента}

На странице удаления абонента (рис. \ref{fig:subscriber-deleting-form}) находится лишь форма абонентского счёта, а также кнопка для удаления абонента из системы по данному абонентскому счёту.
\img{subscriber-deleting-form}{0.5}{Форма удаления абонента}

Если авторизоваться в системе как абонент, то произойдет переход к следующей форме -- меню взаимодействия с системой для абонента (рис. \ref{fig:subscriber-menu-form}). На данной странице абонент может выбрать 2 возможных типа взаимодействия: посмотреть данные о себе и подключить тариф.
\img{subscriber-menu-form}{0.5}{Меню взаимодействия с системой для абонента}

На странице подключения тарифа (рис. \ref{fig:tariff-connecting-form}) находится список всех существующих тарифов с краткой информацией и рядом расположенными кнопками для подключения соответствующего тарифа. Также на странице расположен поиск по названию тарифа.
\img{tariff-connecting-form}{0.5}{Форма подключения тарифа}


\subsection{Описание отчётов}

С главной страницы можно перейти к просмотру всех актуальных тарифов сотового оператора (рис. \ref{fig:tariffs}). Можно увидеть следующую информацию о каждом тарифе: название, абонентская плата, интернет трафик, количество минут и количество SMS.
\img{tariffs}{0.5}{Отчёт об актуальных тарифах сотового оператора}

Будучи авторизованным как продавец-консультант, можно просмотреть все свободные телефонные номера в системе (рис. \ref{fig:free-phone-numbers}). Отображается лишь список свободных номеров.
\img{free-phone-numbers}{0.5}{Отчёт о свободных телефонных номерах сотового оператора}

Будучи авторизованным как абонент, можно просмотреть информацию о своём аккаунте (рис. \ref{fig:subscriber-info}). Можно просмотреть следующие данные об абоненте: абонентский счёт, баланс на абонентском счёте, телефонный номер, подключённый тариф.
\img{subscriber-info}{0.5}{Отчёт о данных абонентского аккаунта}


\subsection{Описание запросов}


\renewcommand{\arraystretch}{1.5}
\begin{xltabular}[h]{\textwidth}{|p{0.7 \textwidth}|X|}
    \caption{Описание запросов} \\
    \hline
    \textbf{Запрос}                                                                                                                                                                                                       & \textbf{Для чего используется}                                                  \\
    \hline \endhead
    \texttt{SELECT * FROM tariffs}                                                                                                                                                                                        & Получение всей информации о каждом тарифе в системе                             \\ \hline
    \texttt{SELECT login, password, user\_type FROM users INNER JOIN user\_types ON id=user\_type\_id WHERE login='\{логин\}' AND password='\{пароль\}' LIMIT 1}                                                          & Получение данных авторизации пользователя                                       \\ \hline
    \texttt{INSERT INTO tariffs (name, subscription\_fee, internet\_traffic, minutes, sms) VALUES ('\{название\}', \{абонентская плата\}, \{интернет трафик\}, \{количество минут\}, \{количество SMS\})}                 & Добавление нового тарифа в систему                                              \\ \hline
    \texttt{UPDATE tariffs SET \{атрибут 1\}=\{новое значение атрибута 1\}, ..., \{атрибут n\}=\{новое значение атрибута n\} WHERE name='\{название\}'}                                                                   & Обновление данных о тарифе                                                      \\ \hline
    \texttt{SELECT COUNT(*) as count FROM clients WHERE passport=\{серия и номер паспорта\} LIMIT 1}                                                                                                                      & Проверка, существует ли клиент с данной серией и номером паспорта               \\ \hline
    \texttt{UPDATE clients SET \{атрибут 1\}=\{новое значение атрибута 1\}, ..., \{атрибут n\}=\{новое значение атрибута n\} WHERE passport='\{серия и номер паспорта\}'}                                                 & Обновление данных о клиенте                                                     \\ \hline
    \texttt{SELECT id FROM places\_of\_registration WHERE place\_of\_registration=\{место прописки\} LIMIT 1}                                                                                                             & Получение кода данного места прописки, если оно существует                      \\ \hline
    \texttt{INSERT INTO places\_of\_registration (place\_of\_registration) values ('\{место прописки\}')}                                                                                                                 & Добавление места прописки в систему                                             \\ \hline
    \texttt{INSERT INTO clients (passport, full\_name, place\_of\_registration\_id, date\_of\_birth) VALUES (\{серия и номер паспорта\}, '\{ФИО\}', \{код места прописки\}, TO\_DATE('\{дата рождения\}', 'dd.mm.yyyy'))} & Добавление клиента в систему                                                    \\ \hline
    \texttt{SELECT COUNT(*) as count FROM subscribers WHERE passport=\{серия и номер паспорта\} AND balance<0}                                                                                                            & Проверка, нет ли задолженности у клиента ни по одному из его абонентских счетов \\ \hline
    \texttt{SELECT COUNT(*) as count FROM subscribers WHERE passport=\{серия и номер паспорта\}}                                                                                                                          & Получение количества зарегистрированных на клиента абонентских счетов           \\ \hline
    \texttt{INSERT INTO subscribers (account, passport) VALUES ('\{абонентский счёт\}', \{серия и номер паспорта\})}                                                                                                      & Добавление абонента в систему                                                   \\ \hline
    \texttt{UPDATE phone\_numbers SET subscriber\_account='\{абонентский счёт\}' WHERE phone\_number=\{телефонный номер\}}                                                                                                & Регистрация абонентского счёта на данный телефонный номер                       \\ \hline
    \texttt{INSERT INTO users (login, password, user\_type\_id) VALUES ('\{телефонный номер\}', '\{абонентский счёт\}', (SELECT id FROM user\_types WHERE user\_type='Абонент'))}                                         & Добавление данных авторизации абонента по умолчанию                             \\ \hline
    \texttt{SELECT COUNT(*) as count FROM subscribers WHERE account='\{абонентский счёт\}' AND balance<0}                                                                                                                 & Проверка, нет ли задолженности по данному абонентскому счёту                    \\ \hline
    \texttt{DELETE FROM subscribers WHERE account='\{абонентский счёт\}'}                                                                                                                                                 & Удаление абонента из системы                                                    \\ \hline
    \texttt{DELETE FROM users WHERE login='\{телефонный номер\}'}                                                                                                                                                         & Удаление данных авторизации абонента                                            \\ \hline
    \texttt{SELECT phone\_number FROM phone\_numbers WHERE subscriber\_account IS NULL}                                                                                                                                   & Получение всех свободных телефонных номеров сотового оператора                  \\ \hline
    \texttt{UPDATE subscribers SET connected\_tariff='\{название тарифа\}'  WHERE account='\{абонентский счёт\}'}                                                                                                         & Подключение данного тарифа на определённый абонентский счёт                     \\ \hline
    \texttt{SELECT COUNT(*) as count FROM subscribers INNER JOIN tariffs ON name='\{название тарифа\}' AND balance>=subscription\_fee WHERE account='\{абонентский счёт\}'}                                               & Проверка, хватает ли денег на балансе для подключения данного тарифа            \\ \hline
    \texttt{SELECT account, balance, phone\_number, connected\_tariff FROM subscribers INNER JOIN phone\_numbers ON subscriber\_account=account WHERE account='\{абонентский счёт\}'}                                     & Получение данных об абоненте с данным абонентским счётом                        \\ \hline
\end{xltabular}


\subsection{Описание диалогов}


При попытке неверно ввести логин и пароль появляется следующее диалоговое окно (рис. \ref{fig:failed-auth}):
\img{failed-auth}{0.5}{Диалог при неверном вводе логина или пароля в систему}

При попытке добавить тариф с уже существующим названием появится следующее диалоговое окно (рис. \ref{fig:invalid-tariff-name-key}):
\img{invalid-tariff-name-key}{0.5}{Диалог при добавлении существующего тарифа в систему}

При попытке добавления или изменения тарифа после неправильного ввода в поле появляется следующее диалоговое окно (рис. \ref{fig:invalid-tariff-attribute}):
\img{invalid-tariff-attribute}{0.5}{Диалог при неправильном вводе атрибутов тарифа}

При успешном добавлении тарифа появляется следующее диалоговое окно (рис. \ref{fig:success-tariff-adding}):
\img{success-tariff-adding}{0.5}{Диалог при успешном добавлении тарифа в систему}

При успешном изменении тарифа появляется следующее диалоговое окно (рис. \ref{fig:success-tariff-editing}):
\img{success-tariff-editing}{0.5}{Диалог при успешном изменении тарифа в систему}

При поиске серии и номера паспорта, которых нет в системе, появится следующее диалоговое окно (\ref{fig:passport-not-found}):
\img{passport-not-found}{0.5}{Диалог при безуспешном поиске клиента по серии и номеру паспорта}

При поиске серии и номера паспорта, которые существуют в системе, появится следующее диалоговое окно (\ref{fig:passport-found}):
\img{passport-found}{0.5}{Диалог при успешном поиске клиента по серии и номеру паспорта}

При попытке добавления или изменения данных о клиенте после неправильного ввода в поле появляется следующее диалоговое окно (рис. \ref{fig:invalid-client-attribute}):
\img{invalid-client-attribute}{0.5}{Диалог при неправильном вводе атрибутов клиента}

При успешном добавлении данных о клиенте появляется следующее диалоговое окно (рис. \ref{fig:success-client-adding}):
\img{success-client-adding}{0.5}{Диалог при успешном добавлении данных о клиенте в систему}

При успешном изменении данных о клиенте появляется следующее диалоговое окно (рис. \ref{fig:success-client-editing}):
\img{success-client-editing}{0.5}{Диалог при успешном изменении данных о клиенте в системе}

При попытке добавления абонента, данных о котором нет в системе, появляется следующее диалоговое окно (рис. \ref{fig:subscriber-info-not-found}):
\img{subscriber-info-not-found}{0.5}{Диалог при безуспешном поиске данных об абоненте}

При попытке добавления абонента, данных о котором есть в системе, появляется следующее диалоговое окно (рис. \ref{fig:subscriber-info-found}):
\img{subscriber-info-found}{0.5}{Диалог при успешном поиске данных об абоненте}

При попытке добавления абонента, данные которого уже зарегистрированы на 5 абонентских счетов, появляется следующее диалоговое окно (рис. \ref{fig:too-many-accounts}):
\img{too-many-accounts}{0.5}{Диалог при добавлении абонента, данные которого уже зарегистрированы на 5 абонентских счетов}

При попытке добавления абонента, по абонентским счетам которого имеется хотя бы одна задолженность, появляется следующее диалоговое окно (рис. \ref{fig:subscriber-debts}):
\img{subscriber-debts}{0.5}{Диалог при добавлении абонента, по абонентским счетам которого имеется хотя бы одна задолженность}

При попытке добавления абонента с уже существующим абонентским счетом появляется следующее диалоговое окно (рис. \ref{fig:invalid-subscriber-account}):
\img{invalid-subscriber-account}{0.5}{Диалог при абонента с уже существующим абонентским счетом}

При успешном добавлении тарифа появляется следующее диалоговое окно (рис. \ref{fig:success-subscriber-adding}):
\img{success-subscriber-adding}{0.5}{Диалог при успешном добавлении абонента в систему}

При попытке удаления абонента, абонентского счёта которого не существует, появляется следующее диалоговое окно (рис. \ref{fig:subscriber-account-not-found}):
\img{subscriber-account-not-found}{0.5}{Диалог при удалении абонента, абонентского счёта которого не существует}

При попытке удаления абонента, абонентский счёт которого существует, появляется следующее диалоговое окно (рис. \ref{fig:subscriber-account-found}):
\img{subscriber-account-found}{0.5}{Диалог при удалении абонента, абонентский счёт которого существует}

При попытке удаления абонента, по абонентскому счёту которого имеется задолженность, появляется следующее диалоговое окно (рис. \ref{fig:subscriber-account-debt}):
\img{subscriber-account-debt}{0.5}{Диалог при удалении абонента, по абонентскому счёту которого имеется задолженность}

При попытке подключения тарифа, баланс абонентского счёта которого меньше, чем стоимость абонентской платы соответствующего тарифа, появляется следующее диалоговое окно (рис. \ref{fig:subscriber-low-balance}):
\img{subscriber-low-balance}{0.5}{Диалог при подключении тарифа, баланс абонентского счёта которого меньше, чем стоимость абонентской платы соответствующего тарифа}

При успешном подключении тарифа появляется следующее диалоговое окно (рис. \ref{fig:success-tariff-connecting}):
\img{success-tariff-connecting}{0.5}{Диалог при успешном подключении тарифа}

При попытке получения списка актуальных тарифов в то время, как их не существует, появляется следующее диалоговое окно (рис. \ref{fig:no-tariffs}):
\img{no-tariffs}{0.5}{Диалог при просмотре актуальных тарифов в то время, как их не существует}

При попытке получения списка свободных телефонных номеров в то время, как их не существует, появляется следующее диалоговое окно (рис. \ref{fig:no-phone-numbers}):
\img{no-phone-numbers}{0.5}{Диалог при просмотре свободных телефонных номеро в то время, как их не существует}