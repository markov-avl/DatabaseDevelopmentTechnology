\section{Анализ предметной области}\label{sec:domain-analysis}


\subsection{Основные понятия и пользователи}


Оператор сотовой связи -- это компания, предоставляющая услуги сотовой связи для сотовых телефонов своих клиентов. Клиент -- это лицо, заинтересованное в получении услуг данной компании. Между компанией и клиентом может быть заключен абонемент -- договор, по которому одна сторона, называемая абонентом, имеет право периодически требовать оказания определённых услуг от компании в течение срока действия договора абонемента (в данном случае только клиент может быть абонентом).

Сотовый оператор владеет множеством телефонных номеров вида \texttt{7\{К\}\{АН\}}, где \texttt{К} -- код города/оператора, состоящий из 3-х цифр, а \texttt{АН} -- абонентский номер, состоящий из 7-и цифр. Эти телефонные номера принадлежат только одному сотовому оператору и не могут быть использованы другими операторами.

Оператор сотовой связи также обладает такими работниками как <<продавец-консультант>>, которые работают в салонах связи -- в торговых точках, предоставляющих комплексные услуги сотовой связи. Основной их работой является обслуживание клиентов.

Продавцы-консультанты занимаются продажей SIM-карт сотового оператора. SIM-карта -- это идентификационный электронный модуль мобильной связи сотового оператора для абонента, который и позволяет ему использовать услуги компании.

Продавцы-консультанты знают всю актуальную информацию о тарифах, которые распространяет сотовый оператор. Тариф -- это включаемые услуги и ставки оплаты за эти же услуги, предоставляемые компанией. Тариф характеризуется следующими данными:
% возможно нужно написать, кто создает тарифы 
\begin{itemize}
    \item название;
    \item абонентская плата (руб.);
    \item интернет трафик (ГБ);
    \item количество минут (мин.);
    \item количество SMS (шт.).
\end{itemize}

Каждый тариф предполагает использование услуг, включаемых в этот тариф, и ежемесячную их оплату, равную соответствующей абонентской плате. Неоплата ведёт к отключению абонента от услуг и его задолженности компании. Далее представлены некоторые актуальные тарифы сотового оператора.

\begin{table}[H]
    \label{tab:actual-tariffs}
    \caption{Актуальные тарифы}
    \setlength{\parskip}{1.0ex}
    \renewcommand{\arraystretch}{1.5}
    \renewcommand{\tabularxcolumn}[1]{m{#1}}
    \begin{tabularx}{\textwidth}{|X|C|C|C|C|}
        \hline
                                           & \textbf{Стандарт-ный} & \textbf{Супер SMART} & \textbf{Всегда в сети} & \textbf{Мой бизнес} \\ \hline
        \textbf{Абонент-ская плата (руб.)} & 150                   & 600                  & 550                    & 1000                \\ \hline
        \textbf{Интернет трафик (ГБ)}      & 0,1                   & 15                   & $\infty$               & 35                  \\ \hline
        \textbf{Количество минут (мин.)}   & 50                    & 600                  & 50                     & $\infty$            \\ \hline
        \textbf{Количество SMS (шт.)}      & 50                    & 600                  & 50                     & $\infty$            \\ \hline
    \end{tabularx}
\end{table}

% не хватает сноски на символ бесконечности - что он означает, но я не знаю, как правильно сделать эту сноску

% возможно нужно описать, как работают тарифы (как происходит оплата, что такое задолженность)

Клиент может обратиться к продавцу-консультанту одного из салонов за оказанием следующих услуг:
\begin{itemize}
    \item покупка SIM-карты (регистрация SIM-карты на имя клиента);
    \item расторжение договора абонемента;
    \item предоставление актуальной информации о тарифах сотового оператора.
\end{itemize}


\subsection{Как происходит покупка SIM-карты}


Покупка SIM-карты проходит поэтапно. Первый этап -- этап проверки выполнения следующих условий со стороны клиента:
\begin{itemize}
    \item обладает паспортом совершеннолетнего;
    \item имеет менее 5-и зарегистрированных на своё имя SIM-карт;
    \item (рассматривается, если уже обладает хотя бы одной SIM-картой, выпущенной данным сотовым оператором) не имеет задолженности ни по одному из тарифов, зарегистрированных на SIM-картах клиента.
\end{itemize}

Если хотя бы одно условие проверки не выполняется, то клиент не может купить SIM-карту. Если же все условия выполняются, то осуществляется переход на следующий этап -- выбор телефонного номера клиентом. Продавец-консультант предоставляет один из свободных телефонных номеров -- номера, на которые в данный момент никто не зарегистрирован. Клиент выбирает наиболее понравившийся телефонный номер, и теперь может перейти к следующему этапу покупки -- подписание договора абонемента.

В договоре заполняются следующие поля:
\begin{itemize}
    \item телефонный номер, выбранный клиентом;
    \item абонентский номер (последние 7 цифр в телефонном номере);
    \item ФИО клиента;
    \item дата рождения клиента;
    \item номер паспорта клиента;
    \item фактический адрес проживания клиента;
    \item место прописки клиента;
    \item расчётный счёт, состоящий из 20 цифр (уже привязан к выдаваемой SIM-карте);
    \item дата заключения договора вида \texttt{\{ДД\}.\{ММ\}.\{ГГГГ\}}.
\end{itemize}

После его успешного заключения клиент становится абонентом данного оператора сотовой связи. Если абонент впервые заключает договор с данным оператором сотовой связи, то продавец-консультант вносит в систему следующие данные об абоненте: ФИО, дата рождения, номер паспорта, фактический адрес проживания, место прописки. Если же абонент уже заключал договор с данным сотовым оператором и некоторые из его данных изменились -- продавец-консультант изменяет эти данные в системе на актуальные.

Затем продавец-консультант добавляет данные договора в систему, и выбранный абонентом телефонный номер автоматически помечается как занятый. Только после всех этих этапов продавец-консультант выдаёт SIM-карту абоненту.


\subsection{Как происходит расторжение договора абонемента}


Абонент имеет право прекратить договорные отношения между собой и компанией сотового оператора, для этого он может обратиться к продавцу-консультанту. Расторжение договора абонемента может производиться лишь при выполнении следующих условий со стороны абонента:
\begin{itemize}
    \item должен знать свой абонентский номер;
    \item должен предъявить любой документ, удостоверяющий его личность;
    \item не имеет задолженности по тарифу, зарегистрированному на SIM-карте, привязанной к данному абонентскому счёту.
\end{itemize} 

Если все эти условия выполнены, абоненту даётся бланк заявления на расторжение договора ...





% Если абонемент был заключён успешно, клиент получает возможность 



% Каждый такой номер может быть занятым (если существует абонент, владеющий данным телефонным номером) или незанятым (иначе).

% Требуется разработать информационную систему для хранения данных об абонентах оператора сотовой связи. Такую информационную систему будем называть телефонным справочником, где каждая запись должна иметь следующую информацию:
% \begin{itemize}
%     \item телефонный номер вида \texttt{7\{К\}\{АН\}}, где \texttt{К} -- код города/оператора, состоящий из 3-х цифр, а \texttt{АН} -- абонентский номер, состоящий из 7-и цифр;
%     \item ФИО владельца телефонного номера (далее просто владелец); % нормально ли так писать "владелец"?
%     \item дата рождения владельца;
%     \item номер паспорта владельца;
%     \item место прописки владельца.
% \end{itemize}

% Для создания телефонного справочника требуется базу данных телефонных номеров, зарегистрированных для данного сотового оператора. Иными словами, эта база данных должна состоять из телефонных номеров

% Поддержкой данных телефонного справочника в актуальном состоянии занимаются продавцы-операторы -- официальные работники салонов сотовой связи данной компании. Продавец-оператор имеет права на добавление, удаление и обновление записей в телефонном справочнике:
% \begin{enumerate}
%     \item \textbf{Добавление записей.} При обращении к продавцу-оператору одного из салонов, клиент может заключить абонемент между компанией сотовой связи. Заключение абонемента проходит успешно, если клиент предоставляет свой паспорт совершеннолетнего, указывает свой фактический адрес проживания, а также выполняет следующие условия:
%     \begin{itemize}
%         \item имеет менее 5-и зарегистрированных на себя абонементов;
%         \item не имеет задолженности (неуплаты) по одному или нескольким уже зарегистрированным на себя абонементам.
%     \end{itemize}
%     Если клиент заключает договор с компанией впервые, то продавец-оператор заводит нового абонента в телефонном справочнике, добавляя фактический адрес проживания клиента и некоторые его паспортные данные: имя, фамилия, отчество, дата рождения, номер паспорта, место прописки.

%     % возможно надо написать, что оператор уже имеет в системе несколько телефонных номеров, которые может предлагать своим клиентам
%     При заключении нового договора, абонент получает уникальный телефонный номер (на свой выбор, из имеющихся в системе как свободные, то есть ни к кому не привязанные), который привязывается к зарегистрированному пользователю (клиенту) и вносится в телефонный справочник продавцом-оператором.
%     \item \textbf{Удаление записей.} Любой абонент, не имеющий задолженности по своему абонементу, может его расторгнуть. Тогда 
% \end{enumerate}

% Любое изменение в телефонном справочнике основывается на реальном документе, подтверждающем то или иное действие.

% % Телефонный справочник может быть использован как в сфере работы предприятий, предоставляющих услуги телефонной связи, так и в других целях. Например, его можно использовать в работе справочной службы, когда любой человек по фамилии или по номеру телефона абонента может узнать его адрес.

% За получением дополнительной информации об услугах данного оператора сотовой связи клиент может обратиться к консультанту -- официальному работнику салона дилеров данной компании. Клиент имеет право получить любую информацию о тарифах данного сотового оператора -- 

% Далее представлены некоторые актуальные тарифы сотового оператора.

% % TODO: сделать сноску для символа бесконечности, что это означает безлимит

% % убрать "так же"
% % описать условия заключения договора
% % описать, откуда берутся данные
% Чтобы стать абонентом услуг оператора сотовой связи, клиент должен заключить договор между компанией. Сделать это можно так же у консультанта после обсуждения всех условий будущего договора. В договоре клиент указывает свои следующие данные:
% \begin{itemize}
%     \item ФИО;
%     \item дата рождения;
%     \item номер паспорта;
%     \item фактический адрес проживания;
%     \item место прописки.
% \end{itemize}

% После успешного заключения договора и оплаты за новый телефонный номер, консультант выдает клиенту SIM-карту -- идентификационный электронный модуль мобильной связи абонента, к которому привязан выбранный клиентом телефонный номер.

% Также клиент, при оплате за новый телефонный номер, уже может выбрать тарифный план для данной него, и тогда некоторая сумма оплаты перейдёт на его баланс. Стоимость такого комплекта зависит от выбранного тарифного плана и вида телефонного номера. Существует 2 вида телефонных номеров:
% \begin{enumerate}
%     \item \textbf{Кривой номер} --- обычный номер, который не имеет особых запоминающихся чередований цифр в своей записи.
%     \item \textbf{Красивый номер} --- необычный номер, который может обладать длинными последовательностями из повторяющихся цифр и особыми запоминающимися чередованиями цифр в своей записи.
% \end{enumerate}

% Далее представлен пример расценок за получение новой SIM-карты в рублях.

% \begin{table}[H]
%     \label{tab:costs-example}
%     \caption{Примеры стоимостей комплектов <<Тарифный план + номер>>}
% 	\setlength{\parskip}{1.0ex}
% 	\renewcommand{\arraystretch}{1.5}
%     \renewcommand{\tabularxcolumn}[1]{m{#1}}
%     \begin{tabularx}{\textwidth}{|X|c|c|c|}
%         \hline
%                                 & \textbf{Без тарифа} & \textbf{SMART} & \textbf{Супер SMART} \\ \hline
%         \textbf{Кривой номер}   & 200                 & 450            & 700                  \\ \hline
%         \textbf{Красивый номер} & 1500                & 1750           & 2000                 \\ \hline
%     \end{tabularx}
% \end{table}

% После проведения всех операций, связанных с выдачей нового телефонного номера, консультант заносит в систему всю информацию абонента из договора, а также данные его новой SIM-карты:
% \begin{itemize}
%     \item уникальный телефонный номер;
%     \item вид телефонного номера;
%     \item код текущего тарифа (может и не быть);
%     \item текущий баланс, высчитываемый по таблице \ref{tab:costs-example}. % нужно расписать по какой именно формуле высчитывается?
%     % TODO: почему-то неправильная нумерация таблицы отображается
% \end{itemize}

% % TODO: написать, что такое абонентский аккаунт
% В свою очередь, абоненты данного оператора имеют право узнать информацию не только о существующих тарифах компании, но и узнать всю информацию о состоянии своего абонентского аккаунта при предъявлении паспорта, на номер которого он был создан.

% % в итоге написать профессионалов
