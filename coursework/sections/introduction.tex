\section*{Введение}\label{sec:introduction}
\addcontentsline{toc}{section}{Введение}


\setlist[enumerate, 1]{align=left, leftmargin=0cm, labelindent=1cm, listparindent=1cm, labelwidth=*, itemindent=2cm}


На данный момент огромное количество задач сводятся к получению, обработке и хранению огромных объемов информации. Автоматизация данных действий с помощью алгоритмических языков не подходит. Поэтому на передний план выходят базы данных, которые позволяют обеспечить многоаспектный доступ к совокупности взаимосвязанных данных и централизацию управления данными. Тем самым устраняется избыточность хранимых данных.

Также у баз данных существуют математически проработанные модели данных, что позволяет избежать аномалий добавления, удаления и изменение информации. Что позволяет поддерживать достоверность хранимых данных, а также обеспечить быстроту выполнения операций над данными.

Базы данных, например, используются сотовыми операторами для хранения информации о своих телефонных номерах, тарифах и абонентах. Базы данных также позволяют автоматизировать некоторые процессы, происходящие в данной предметной области, из-за чего они становятся просто незаменимы.

Цель курсового проекта: разработать и реализовать проект системы базы данных «Онлайн кинотеатр» с использованием реляционной СУБД MS Access.

Для достижения поставленной цели необходимо решить следующие задачи:
\begin{enumerate}
    \item Провести анализ предметной области.
    \item Построить инфологическую модель.
    \item Построить даталогическую модель.
    \item Провести обзор СУБД Access и реализовать БД.
    \item Провести тестирование БД.
\end{enumerate}


\newpage