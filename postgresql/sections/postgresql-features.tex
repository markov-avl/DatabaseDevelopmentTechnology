\section{Особенности PostgreSQL}\label{sec:postgresql-features}

\subsection{Функции}\label{subsec:functions}

Функции являются блоками кода, исполняемыми на сервере, а не на клиенте БД. Хотя они могут писаться на чистом SQL, реализация дополнительной логики, например, условных переходов и циклов, выходит за рамки собственно SQL и требует использования некоторых языковых расширений. Функции могут писаться с использованием одного из следующих языков:
\begin{itemize}
    \item Встроенный процедурный язык PL/pgSQL, во многом аналогичный языку PL/SQL, используемому в СУБД Oracle;
    \item Скриптовые языки -- PL/Lua, PL/LOLCODE, PL/Perl, PL/PHP, PL/Python, PL/Ruby, PL/sh, PL/Tcl и PL/Scheme;
    \item Классические языки -- C, C++, Java (через модуль PL/Java);
    \item Статистический язык R (через модуль PL/R).
\end{itemize}

PostgreSQL допускает использование функций, возвращающих набор записей, который далее можно использовать так же, как и результат выполнения обычного запроса.
Функции могут выполняться как с правами их создателя, так и с правами текущего пользователя.

Иногда функции отождествляются с хранимыми процедурами, однако между этими понятиями есть различие. С девятой версии возможно написание автономных блоков, которые позволяют выполнять код на процедурных языках без написания функций, непосредственно в клиенте.

\subsection{Триггеры}\label{subsec:triggers}

Триггеры определяются как функции, инициируемые DML-операциями. Например, операция \texttt{INSERT} может запускать триггер, проверяющий добавленную запись на соответствия определённым условиям. При написании функций для триггеров могут использоваться различные языки программирования (см. выше). % нужна ссылка

Триггеры ассоциируются с таблицами. Множественные триггеры выполняются в алфавитном порядке.

\subsection{Механизм правил}\label{subsec:rules}

Механизм правил (англ. \textit{rules}) представляет собой механизм создания пользовательских обработчиков не только DML-операций, но и операции выборки. Основное отличие от механизма триггеров заключается в том, что правила срабатывают на этапе разбора запроса, до выбора оптимального плана выполнения и самого процесса выполнения. Правила позволяют переопределять поведение системы при выполнении SQL-операции к таблице. Хорошим примером является реализация механизма представлений (англ. \textit{views}): при создании представления создается правило, которое определяет, что вместо выполнения операции выборки к представлению система должна выполнять операцию выборки к базовой таблице / таблицам с учетом условий выборки, лежащих в основе определения представления. Для создания представлений, поддерживающих операции обновления, правила для операций вставки, изменения и удаления строк должны быть определены пользователем.

\subsection{Индексы}\label{subsec:indexes}

В PostgreSQL имеется поддержка индексов следующих типов: B-дерево, хэш, R-дерево, GiST, GIN. При необходимости можно создавать новые типы индексов, хотя это далеко не тривиальный процесс. Индексы в PostgreSQL обладают следующими свойствами:
\begin{itemize}
    \item возможен просмотр индекса не только в прямом, но и в обратном порядке – создание отдельного индекса для работы конструкции \texttt{ORDER BY ... DESC} не нужно;
    \item возможно создание индекса над несколькими столбцами таблицы, в том числе над столбцами различных типов данных;
    \item индексы могут быть функциональными, то есть строиться не на базе набора значений некоего столбца / столбцов, а на базе набора значений функции от набора значений;
    \item индексы могут быть частичными, то есть строиться только по части таблицы (по некоторой её проекции); в некоторых случаях это помогает создавать намного более компактные индексы или достигать улучшения производительности за счёт использования разных типов индексов для разных (например, с точки зрения частоты обновления) частей таблицы;
    \item планировщик запросов может использовать несколько индексов одновременно для выполнения сложных запросов.
\end{itemize}

\subsection{Многоверсионность (MVCC)}\label{subsec:mvcc}

PostgreSQL поддерживает одновременную модификацию БД несколькими пользователями с помощью механизма MultiVersion Concurrency Control (MVCC). Благодаря этому соблюдаются требования ACID, и практически отпадает нужда в блокировках чтения.

\subsection{Расширение}\label{subsec:widening}

Расширение PostgreSQL для собственных нужд возможно практически в любом аспекте. Есть возможность добавлять собственные преобразования типов, типы данных, домены (пользовательские типы с изначально наложенными ограничениями), функции (включая агрегатные), индексы, операторы (включая переопределение уже существующих) и процедурные языки.

\subsection{Наследование}\label{subsec:inheritance}

Наследование в PostgreSQL реализовано на уровне таблиц. Таблицы могут наследовать характеристики и наборы полей от других таблиц (родительских). При этом данные, добавленные в порождённую таблицу, автоматически будут участвовать (если это не указано отдельно) в запросах к родительской таблице.

\subsection{Пользовательские объекты}\label{subsec:user-objects}

PostgreSQL может быть расширен пользователем для собственных нужд практически в любом аспекте. Есть возможность добавлять собственные:
\begin{itemize}
    \item преобразования типов;
    \item типы данных;
    \item домены (пользовательские типы с изначально наложенными ограничениями);
    \item функции (включая агрегатные);
    \item индексы;
    \item операторы (включая переопределение уже существующих);
    \item процедурные языки.\cite{web-creator}
\end{itemize}