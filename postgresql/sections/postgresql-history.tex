\section{Краткая история PostgreSQL}\label{sec:postgresql-history}

\subsection{POSTGRES}\label{subsec:postgres}

Реализация реляционной СУБД POSTGRES началась в 1986. Затем вышло несколько версий Postgres. Первая система заработала в 1987 и была продемонстрирована в 1988.

СУБД POSTGRES была использована для реализации многих различных исследований и написания приложений. Сюда вошли: система анализа финансовых данных, пакет мониторинга производительности реактивных двигателей, база данных перемещений астероидов, база данных медицинской информации и несколько географических информационных систем. POSTGRES также использовалась как средство обучения в нескольких университетах.

Размер сообщества пользователей этого продукта удвоился в 1993 году. Стало весьма очевидно, что обслуживание прототипа кода и его поддержка занимают гораздо больше времени, чем сами исследования в области баз данных. Пытаясь снизить нагрузку, связанную с поддержкой, проект Беркли POSTGRES официально прекратил своё существование с выходом версии 4.2.

\subsection{Postgres95}\label{subsec:postgres95}

В 1994 Эндрю Ю (Andrew Yu) и Джолли Чен (Jolly Chen) дали проекту POSTGRES второе дыхание, добавив в него интерпретатор языка SQL и выложив в сеть, чтобы найти свой собственный путь в мире продуктов с открытым исходным кодом.

Postgres95 был полностью приведён к стандарту ANSI C и сократил свой размер на 25\%. Были внесены многие внутренние изменения, которые увеличили производительность и обслуживаемость кода. Postgres95 версий 1.0.x был быстрее на 30–50\% по сравнению с POSTGRES, Version 4.2.

\subsection{PostgreSQL}\label{subsec:postgresql}

В 1996 году было решено, что имя «Postgres95» не соответствует настоящему времени. Разработчики выбрали новое имя PostgreSQL чтобы подчеркнуть отличие от оригинального POSTGRES и выход множества версий с поддержкой SQL.

При разработке Postgres95 акцент ставился на обнаружение и понимание существующих проблем в коде сервера. В PostgreSQL акцент сместился на расширение возможностей и совместимости при продолжении работы во всех других областях.\cite{postgresql-12}
